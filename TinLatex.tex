\NeedsTeXFormat{LaTeX2e}

\documentclass[11pt]{TinLatex} 

\textwidth15.4cm

\textheight21.8cm

\oddsidemargin0.0cm 

\evensidemargin0.0cm

\setlength{\topmargin}{0cm}
\setlength{\headheight}{0.4cm}
\setlength{\headsep}{.5cm}



%\topmargin

\usepackage{graphicx}
\usepackage{epstopdf}
\usepackage{xspace}

\usepackage{amsmath,amsxtra,amssymb,latexsym, amscd}

\usepackage[mathscr]{eucal}
\usepackage{hyperref}
\hypersetup{
	colorlinks=false,
	pdfborder={0 0 0}
}


%\renewcommand\refname{\normalsize \centerline{ REFERENCES}}

\input setbmp
\input seteps
\input setwmf
\input setps
\input dih.tex

\font\elevent=cmr12 at 11pt
%\parskip=2pt
\font\it=cmti12 at 11pt
\font\bf=cmbx12 at 11pt
\font\cv=cmr10 scaled \magstephalf
\advance\hoffset0.4cm
%\font\bfv=cmbx10 %scaled \magstephalf
\font\itv=cmti10 %scaled \magstephalf
\font\bfh=cmssbx10 scaled 1400
\font\bfl=cmssbx10 scaled \magstep1
\font\beo=cmr10 scaled\magstep1
\font\chu=cmr8
\font\dsl=cmitt10
\font\cvv=cmr10 scaled \magstephalf
\font\be=cmr6
%%%%%%%%%%
\font\itn=cmti10
\font\bfn=cmbx10
\font\bfitt = cmbxti10 
\font\cn=cmr10
%%%%%%%%%%
\font\itsl=cmsl9
\font\nn=cmsl8
\font\go=eufm10
\font\gotic=eufm8
\font\goticc=eufm6
%%%%%%%%%%%%
\def\vs{\vskip-0.4cm}
\def\vhs{\vskip-0.18cm}
\def\vss{\vskip-0.7cm}
\def\vt{\vskip0.5cm}
\def\vl{\vskip0.3cm}
\def\vlv{\vskip0.27cm}
\def\vv{\vskip0.25cm}%0.25
\def\vvb{\vskip0.21cm}
\def\vb{\vskip0.17cm}%.17
\def\vn{\vskip0.11cm}
\def\vnn{\vskip0.05cm}
\def\n{\noindent}
\def\ce{\centerline}
\def\dis{\displaystyle}
\def\ed{\end{document}}
\def\vuong{\raise-0.16cm\hbox{$^\blacksquare$}}

% Add a period to the end of an abbreviation unless there's one
% already, then \xspace.
\makeatletter
\DeclareRobustCommand\onedot{\futurelet\@let@token\@onedot}
\def\@onedot{\ifx\@let@token.\else.\null\fi\xspace}

\def\eg{e.g\onedot} \def\Eg{E.g\onedot}
\def\ie{i.e\onedot} \def\Ie{I.e\onedot}
\def\cf{c.f\onedot} \def\Cf{C.f\onedot}
\def\etc{etc\onedot} \def\vs{vs\onedot}
\def\wrt{w.r.t\onedot} \def\dof{d.o.f\onedot}
\def\etal{et al\onedot}
\makeatother

\newcommand\blankfootnote[1]{%
	\begingroup
		\renewcommand\thefootnote{}\footnote{#1}%
		\addtocounter{footnote}{-1}%
	\endgroup
}
    